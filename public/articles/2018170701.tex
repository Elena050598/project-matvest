\documentclass[14pt,Diplom]{diplomwork}

\usepackage{fancyvrb}
\usepackage{longtable}
\usepackage{amsmath}% для знака системы 
\renewcommand{\theFancyVerbLine}{\footnotesize\arabic{FancyVerbLine}}


\newcommand{\alert}[1]{{\color{red}#1}}
\sloppy

\date{2020}
\author{студент МКб-4301-51-00}{СВОЕ ИМЯ}

\napravlenie{02.03.01}{Математика и компьютерные науки,\\ профиль <<Математические основы компьютерных наук>>}
\advisor{к. ф.-м. н., доцент}{ИМЯ НАУЧНОГО РУКОВОДИТЕЛЯ}
\kafedra{фундаментальной математики}{Е.\,М.~Вечтомов}
\department{компьютерных и физико-математических наук}{Н.\,А.~Бушмелева}
\institute{математики и информационных систем}

\title{ТЕМА}



\begin{document}

\maketitle
\newpage

\tableofcontents

\Chapter{Введение}

В связи с огромной скоростью развития сети Интернет все большее количество людей использует Интернет-технологии и ресурсы, расположенные в сети. Поэтому неудивительно, что основным средством рекламы, дающим возможность в области поиска и привлечения клиентов, в настоящее время является сайт. Сайт --- это совокупность веб документов, связанных между собой общей тематической направленностью, единым дизайном и ссылками. Сайт представляет физическое или юридическое лицо в глобальной сети Интернет, является одним из современных средств передачи информации, инструментом общения между тем, кто предоставляет какой-либо продукт или  услугу,и клиентом, также это-наиболее распространенный способ представления общей информации о каком-либо продукте. Профессионально построенный сайт повышает престиж учреждения, позволяет оперативно оповещать о регулярно обновляемой тематической информации, способен превратить обычного посетителя, зашедшего на веб-страницу, в потенциального клиента.
 
Для привлечения большего числа посетителей на сайт важно, чтобы он имел не только привлекательный современный дизайн, но был удобен для пользователя и выполнял необходимый функционал. Поэтому возникла необходимость написания нового сайта для периодического межвузовского сборника научно-методических работ <<Математический вестник>> педвузов и университетов Волго-Вятского региона. Старый сайт журнала есть, но он был написан на движке, который уже устарел, также устарела и кодировка сайта. К тому же нет мобильной версии сайта, функционал добавлялся по мере необходимости, поэтому он не в полной мере отвечает всем необходимым требованиям, и некоторые функции вовсе отсутствуют. Среди старых возможностей сайта были: регистрация авторов и их статей,присланных ими для публикации в новом номере журнала, ввод их анкетных данных, публикация каталога уже изданных номеров журнала, а также других книг кафедры фундаментальной математики. Дизайн сайта тоже необходимо обновить, сделать его современным и адаптивным. 

В связи с этим перед мною была поставлена цель написать новый сайт журнала, отвечающий современным технологиям по веб-разработке, с сохранением старого функционала и добавлением нового. 

В соответствии с поставленной целью в работе определены следующие задачи:
\begin{itemize}
\item изучить старый сайт, выявить его недостатки;
\item выбрать способ написания сайта;
\item изучить документацию выбранных фреймворков(laravel, bootstrap);
\item изучить необходимые языки программирования;
\item создать структуру базы данных;
\item разработать дизайн;
\item сверстать страницы будущего сайта;
\item разработать программную часть;
\item провести тестирование;
\item разместить сайт в сети Интернет;
\end{itemize}











		

\begin{thebibliography}{99}


\end{thebibliography}

%\APPENDIX


\end{document}
